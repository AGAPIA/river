%%%%%%%%%%%%%%%%%%%%%%%%%%%%%%%%%%%%%%%%%%%%%%%%%%%%%%%%%%%%%%%%%%%%%%
% writeLaTeX Example: A quick guide to LaTeX
%
% Source: Dave Richeson (divisbyzero.com), Dickinson College
% 
% A one-size-fits-all LaTeX cheat sheet. Kept to two pages, so it 
% can be printed (double-sided) on one piece of paper
% 
% Feel free to distribute this example, but please keep the referral
% to divisbyzero.com
% 
%%%%%%%%%%%%%%%%%%%%%%%%%%%%%%%%%%%%%%%%%%%%%%%%%%%%%%%%%%%%%%%%%%%%%%
% How to use writeLaTeX: 
%
% You edit the source code here on the left, and the preview on the
% right shows you the result within a few seconds.
%
% Bookmark this page and share the URL with your co-authors. They can
% edit at the same time!
%
% You can upload figures, bibliographies, custom classes and
% styles using the files menu.
%
% If you're new to LaTeX, the wikibook is a great place to start:
% http://en.wikibooks.org/wiki/LaTeX
%
%%%%%%%%%%%%%%%%%%%%%%%%%%%%%%%%%%%%%%%%%%%%%%%%%%%%%%%%%%%%%%%%%%%%%%

\documentclass[10pt,landscape]{article}
\usepackage{amssymb,amsmath,amsthm,amsfonts}
\usepackage{multicol,multirow}
\usepackage{calc}
\usepackage{ifthen}
\usepackage[landscape]{geometry}
\usepackage[colorlinks=true,citecolor=blue,linkcolor=blue]{hyperref}
\usepackage{listings}


\ifthenelse{\lengthtest { \paperwidth = 11in}}
    { \geometry{top=.5in,left=.5in,right=.5in,bottom=.5in} }
	{\ifthenelse{ \lengthtest{ \paperwidth = 297mm}}
		{\geometry{top=1cm,left=1cm,right=1cm,bottom=1cm} }
		{\geometry{top=1cm,left=1cm,right=1cm,bottom=1cm} }
	}
\pagestyle{empty}
\makeatletter
\renewcommand{\section}{\@startsection{section}{1}{0mm}%
                                {-1ex plus -.5ex minus -.2ex}%
                                {0.5ex plus .2ex}%x
                                {\normalfont\large\bfseries}}
\renewcommand{\subsection}{\@startsection{subsection}{2}{0mm}%
                                {-1explus -.5ex minus -.2ex}%
                                {0.5ex plus .2ex}%
                                {\normalfont\normalsize\bfseries}}
\renewcommand{\subsubsection}{\@startsection{subsubsection}{3}{0mm}%
                                {-1ex plus -.5ex minus -.2ex}%
                                {1ex plus .2ex}%
                                {\normalfont\small\bfseries}}
\newcommand{\river}{RIVER}
\newcommand{\rr}{\raggedright}
\newcommand{\tn}{\tabularnewline}
\makeatother
\setcounter{secnumdepth}{0}
\setlength{\parindent}{0pt}
\setlength{\parskip}{0pt plus 0.5ex}
% -----------------------------------------------------------------------

\title{Quick Guide to LaTeX}

\begin{document}

\raggedright
\footnotesize

\begin{center}
	\Large{\textbf{{\river} Cheat Sheet}} \\
\end{center}
\begin{multicols}{3}
\setlength{\premulticols}{1pt}
\setlength{\postmulticols}{1pt}
\setlength{\multicolsep}{1pt}
\setlength{\columnsep}{2pt}

\section{What is \river?}
	{\river} spelled backwards stands for \textbf{REV}ersible \textbf{I}ntermediate \textbf{R}epresentation. {\river} is a fixed length extended x86 instruction set. It is designed to be translated to and from x86. {\river} instrumentation helps tracing binaries (no need for program source code) and provides analysis information about each basic block that was executed by target program. Provided details are: module name, instruction pointer offset, execution cost, jump type, jump instruction and number of instructions.

\section{Constant Definitions}

	\subsection{{\river} Modifiers}

	\begin{tabular}{ p{6cm} p{1.3cm} }
		RIVER\_MODIFIER\_NOSEG & 0x00000000\\
		RIVER\_MODIFIER\_ESSEG & 0x00000001\\
		RIVER\_MODIFIER\_CSSEG & 0x00000002\\
		RIVER\_MODIFIER\_SSSEG & 0x00000003\\
		RIVER\_MODIFIER\_DSSEG & 0x00000004\\
		RIVER\_MODIFIER\_FSSEG & 0x00000005\\
		RIVER\_MODIFIER\_GSSEG & 0x00000006\\
							   \\
		RIVER\_MODIFIER\_EXT &	0x00000008\\
		RIVER\_MODIFIER\_O8 & 0x00000010\\
		RIVER\_MODIFIER\_O16 & 0x00000020\\
		RIVER\_MODIFIER\_A16 & 0x00000040\\
		RIVER\_MODIFIER\_LOCK & 0x00000080\\
		RIVER\_MODIFIER\_REP & 0x00000100\\
		RIVER\_MODIFIER\_REPZ & 0x00000200\\
		RIVER\_MODIFIER\_REPNZ & 0x00000400\\
	\end{tabular}

	\subsection{{\river} Family}
	\begin{tabular}{ p{6cm} p{1.3cm} }
		RIVER\_FAMILY\_NATIVE & 0x00000000\\
		RIVER\_FAMILY\_RIVER & 0x00000001\\
		RIVER\_FAMILY\_TRACK & 0x00000002\\
		RIVER\_FAMILY\_PRETRACK & 0x00000003\\
		RIVER\_FAMILY\_PREMETA & 0x00000004\\
		RIVER\_FAMILY\_POSTMETA & 0x00000005\\
		RIVER\_FAMILY\_RIVER\_TRACK & 0x00000006\\

		RIVER\_FAMILY\_FLAG\_METAPROCESSED & 0x00000020\\
		RIVER\_FAMILY\_FLAG\_ORIG\_xSP & 0x00000040\\
		RIVER\_FAMILY\_FLAG\_IGNORE & 0x00000080\\
	\end{tabular}

	\subsection{{\river} Branch Instruction}
	\begin{tabular}{ p{6cm} p{1.3cm} }
		RIVER\_BRANCH\_INSTR\_RET & 0x00001000\\
		RIVER\_BRANCH\_INSTR\_JMP & 0x00002000\\
		RIVER\_BRANCH\_INSTR\_JXX & 0x00004000\\
		RIVER\_BRANCH\_INSTR\_CALL & 0x00008000\\
		RIVER\_BRANCH\_INSTR\_SYSCALL & 0x00010000\\
	\end{tabular}

	\subsection{{\river} Branch Type}
	\begin{tabular}{p{6cm} p{1.3cm} }
		RIVER\_BRANCH\_TYPE\_IMM & 0x00020000\\
		RIVER\_BRANCH\_TYPE\_REG & 0x00040000\\
		RIVER\_BRANCH\_TYPE\_MEM & 0x00080000\\
	\end{tabular}

	\subsection{{\river} Virtual Registers}
	\begin{tabular}{p{6cm} p{1.3cm}}
		RIVER\_REG\_xAX & 0x00\\
		RIVER\_REG\_xCX & 0x01\\
		RIVER\_REG\_xDX & 0x02\\
		RIVER\_REG\_xBX & 0x03\\
		RIVER\_REG\_xSP & 0x04\\
		RIVER\_REG\_xBP & 0x05\\
		RIVER\_REG\_xSI & 0x06\\
		RIVER\_REG\_xDI & 0x07\\
	\end{tabular}

	\subsection{{\river} Virtual Registers Size}
	\begin{tabular}{p{6cm} p{1.3cm}}
		RIVER\_REG\_SZ32 & 0x00\\
		RIVER\_REG\_SZ16 & 0x08\\
		RIVER\_REG\_SZ8\_L & 0x10\\
		RIVER\_REG\_SZ8\_H & 0x18\\
	\end{tabular}

	\subsection{{\river} Segment Registers}
	\begin{tabular}{p{6cm} p{1.3cm}}
		RIVER\_REG\_ES & 0x20\\
		RIVER\_REG\_SEGMENT & 0x20\\
		RIVER\_REG\_CS & 0x21\\
		RIVER\_REG\_SS & 0x22\\
		RIVER\_REG\_DS & 0x23\\
		RIVER\_REG\_FS & 0x24\\
		RIVER\_REG\_GS & 0x25\\
	\end{tabular}

	\subsection{{\river} Control Registers}
	\begin{tabular}{p{6cm} p{1.3cm}}
		RIVER\_REG\_CONTROL & 0x30\\
		RIVER\_REG\_CR0 & 0x30\\
		RIVER\_REG\_CR2 & 0x32\\
		RIVER\_REG\_CR3 & 0x33\\
		RIVER\_REG\_CR4 & 0x34\\
	\end{tabular}

	\subsection{{\river} Debug Registers}
	\begin{tabular}{p{6cm} p{1.3cm}}
		RIVER\_REG\_DEBUG & 0x40\\
		RIVER\_REG\_DR0 & 0x40\\
		RIVER\_REG\_DR1 & 0x41\\
		RIVER\_REG\_DR2 & 0x42\\
		RIVER\_REG\_DR3 & 0x43\\
		RIVER\_REG\_DR4 & 0x44\\
		RIVER\_REG\_DR5 & 0x45\\
		RIVER\_REG\_DR6 & 0x46\\
		RIVER\_REG\_DR7 & 0x47\\
	\end{tabular}

	\subsection{{\river} Other Registers}
	\begin{tabular}{p{6cm} p{1.3cm}}
		RIVER\_REG\_NONE & 0x20\\
	\end{tabular}

	\subsection{RiverAddress components}
	\begin{tabular}{p{6cm} p{1.3cm}}
		RIVER\_ADDR\_DISP8 & 0x01\\
		RIVER\_ADDR\_DISP & 0x02\\
		RIVER\_ADDR\_SCALE & 0x04\\
		RIVER\_ADDR\_BASE & 0x08\\
		RIVER\_ADDR\_INDEX & 0x10\\
		RIVER\_ADDR\_DIRTY & 0x80\\
	\end{tabular}

	\subsection{struct RiverAddress}
	\lstinputlisting[language=C]{code/RiverAddress.h}

	\subsection{{\river} Operand Types}
	\begin{tabular}{p{6cm} p{1.3cm}}
		RIVER\_OPTYPE\_NONE & 0x00\\
		RIVER\_OPTYPE\_IMM & 0x04\\
		RIVER\_OPTYPE\_REG & 0x08\\
		RIVER\_OPTYPE\_MEM & 0x0C\\
		RIVER\_OPTYPE\_ALL & 0x10\\
	\end{tabular}

	\subsection{{\river} Operand Size}
	\begin{tabular}{p{6cm} p{1.3cm}}
		RIVER\_OPSIZE\_32 & 0x00\\
		RIVER\_OPSIZE\_16 & 0x01\\
		RIVER\_OPSIZE\_8 & 0x02\\
	\end{tabular}

	\subsection{{\river} Operation Flags}
	\begin{tabular}{p{6cm} p{1.3cm}}
		RIVER\_SPEC\_IDX\_CF & 0x00\\
		RIVER\_SPEC\_IDX\_PF & 0x01\\
		RIVER\_SPEC\_IDX\_AF & 0x02\\
		RIVER\_SPEC\_IDX\_ZF & 0x03\\
		RIVER\_SPEC\_IDX\_SF & 0x04\\
		RIVER\_SPEC\_IDX\_OF & 0x05\\
		RIVER\_SPEC\_IDX\_DF & 0x06\\
	\end{tabular}


	\subsection{struct RiverOperand}
	\lstinputlisting[language=C]{code/RiverOperand.h}


\section{Resources}
Great symbol look-up site: \href{http://detexify.kirelabs.org/}{Detexify}\\
\href{http://amath.colorado.edu/documentation/LaTeX/Symbols.pdf}{\LaTeX\ Mathematical Symbols}\\
\href{ftp://tug.ctan.org/pub/tex-archive/info/symbols/comprehensive/symbols-letter.pdf}{The Comprehensive \LaTeX\ Symbol List}\\ 
\href{http://mirrors.med.harvard.edu/ctan/info/lshort/english/lshort.pdf}{The Not So Short Introduction to \LaTeX\ 2$\varepsilon$}\\
\href{http://www.tug.org/}{TUG: The \TeX\ Users Group}\\
\href{http://www.ctan.org/}{CTAN: The Comprehensive \TeX\ Archive Network}\\
~\\
\LaTeX\ for the Mac: \href{http://www.tug.org/mactex/}{Mac\TeX}\\
\LaTeX\ for the PC: \href{http://www.texniccenter.org/}{{\TeX}nicCenter} and \href{http://miktex.org/}{MiK\TeX}\\
\LaTeX\ online: \href{http://www.writelatex.com/}{WriteLaTeX}.
\vfill
\hrule
~\\
Dave Richeson, Dickinson College, \href{http://divisbyzero.com/}{http://divisbyzero.com/}
\end{multicols}

\end{document}
